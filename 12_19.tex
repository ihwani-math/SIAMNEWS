%----------------------------------------------------------------------------------------
%	PACKAGES AND THEMES
%----------------------------------------------------------------------------------------
\documentclass[aspectratio=169,xcolor=dvipsnames]{beamer}
\usetheme{SimplePlusAIC}

\usepackage{hyperref}
\usepackage{csquotes}
\newcommand{\code}[1]{\texttt{#1}}
\usepackage{graphicx} % Allows including images
\usepackage{booktabs} % Allows the use of \toprule, \midrule and  \bottomrule in tables
\usepackage{svg} %allows using svg figures
\usepackage{tikz}
\usepackage{makecell}
\newcommand*{\defeq}{\stackrel{\text{def}}{=}}

%Select the Epilogue font (requires luaLatex or XeLaTex compilers)
\usepackage{fontspec}
\setsansfont{Epilogue}[
    Path=./epilogueFont/,
    Scale=0.9,
    Extension = .ttf,
    UprightFont=*-Regular,
    BoldFont=*-Bold,
    ItalicFont=*-Italic,
    BoldItalicFont=*-BoldItalic
    ]

%----------------------------------------------------------------------------------------
%	TITLE PAGE
%----------------------------------------------------------------------------------------

\title[Num. Simulation of Ice Crystal Traj. \& Frag. Dyn.]{Numerical Simulation of Ice Crystal Trajectories and Fragmentation Dynamics Around the XRF1 Fuselage Nose} % The short title appears at the bottom of every slide, the full title is only on the title page
\subtitle{Author: Moussa Diop, Tristan Soubrie, Julien Cliquet, Jean-Mathieu, and Phillippe Villedieu}

\author[Ivan L. Ihwani]{Presented by: Ivan L. Ihwani}
\institute[SIAM NEWS SEPTEMBER 2023]{Department of Mathematics \newline National Central University}
% Your institution as it will appear on the bottom of every slide, maybe shorthand to save space


\date{September 21, 2023} % Date, can be changed to a custom date
%----------------------------------------------------------------------------------------
%	PRESENTATION SLIDES
%----------------------------------------------------------------------------------------

\begin{document}

\begin{frame}[plain]
    % Print the title page as the first slide
    \titlepage
\end{frame}

\begin{frame}{Overview}
    % Throughout your presentation, if you choose to use \section{} and \subsection{} commands, these will automatically be printed on this slide as an overview of your presentation
    \tableofcontents
\end{frame}

%------------------------------------------------
\section{Introduction}
%------------------------------------------------
\begin{frame}{Introduction}
Why this simulation is so important?
    \begin{itemize}
    \item Icing on airplane surfaces often causes a considerable reduction in aerodynamic performance, significant unsteady flow, inconsistent probe measurement, sudden loss of engine thrust, and engine flame out. 
    \item Several aircraft incidents and dramatic accidents that pertain to ice accretion have been recorded since the onset of aviation. 
    \end{itemize}
Water droplets that freeze on solid airplane surfaces typically cause icing. 
\end{frame}

%------------------------------------------------
\begin{frame}{Introduction}
    \begin{itemize}
    \item The ice crystal icing (ICI) has garnered significant research interest in recent years. 
    \item Given the challenge of assessing ICI in ground testing while also meeting flight conditions, the development of the simulation tools is of significant importance.
    \end{itemize}
\end{frame}

%------------------------------------------------
\section{Simulation Tool}

\begin{frame}{Simulation Tool}
    \begin{itemize}
        \item The authors employ a tool, called \href{https://aerospacelab.onera.fr/CEDRE-Software}{\textcolor{blue}{CEDRE}}, in this study. It is a multiphysics CFD code. 
        \item Two solvers are utilized:
        \begin{enumerate}
            \item \textcolor{red}{CHARME}, to simulate the aerodynamic flow field.
            \item \textcolor{red}{SPARTE}, to predict the ice particle trajectories within a Lagrangian approach. 
        \end{enumerate}
        \item This study transpires in a sequential manner, first with CHARME and then SPARTE. 
    \end{itemize}
\end{frame}

%--------------------------------------------------------------------------

\begin{frame}{CHARME: Gas Solver}
    \begin{itemize}
        \item The CHARME solver is used to compute the flow field by solving the Reynolds-averaged Navier-Stokes (RANS) for a gas mixture of two species: air and water vapor. 
        \item Specifically, the authors use \textcolor{blue}{Florian Menter's $k-\omega$ turbulence model} with shear stress transport (SST) correction and perform the time integration with an \textcolor{blue}{implicit first-order Euler scheme}. 
        \item Then carry out the flux calculation with a \textcolor{blue}{second-order monotonic upwind scheme} for the conservation law method and a \textcolor{blue}{Harten-Lax-van Leer contact flux scheme}. 
    \end{itemize}
\end{frame}

\begin{frame}{CHARME: Gas Solver}
    \begin{itemize}
        \item A symmetry plane (XZ) reduces the computational domain to half of the geometry, after which we perform steady computations. The time step was $0.01$ seconds.
    \end{itemize}
\end{frame}


%----------------------------------------------------------------------

\begin{frame}{SPARTE: Lagrangian Particle Solver}
   \begin{itemize}
       \item 
   \end{itemize}
\end{frame}



%------------------------------------------------



%------------------------------------------------
\section{Other Examples}


%------------------------------------------------
\begin{frame}{Other Examples}
    \begin{itemize}
        \item \textcolor{brown}{\href{https://github.com/micjoswig/oscar-notebooks/blob/master/SIAM-News/Hurwitz_Combinatorics.ipynb}{Combinatorics of Hurwitz Numbers}}, tackle problems in enumerative algebraic geometry both in a combinatorial and algorithmic manner.
        \item \textcolor{brown}{\href{https://github.com/micjoswig/oscar-notebooks/blob/master/SIAM-News/Cox_rings_of_blow_ups_of_points.ipynb}{Cox Rings of Point Blow-ups}}, the problem in a geometry of varieties.
        \item \textcolor{brown}{\href{https://github.com/micjoswig/oscar-notebooks/blob/master/SIAM-News/Cox_rings_of_linear_quotients.ipynb}{Cox Rings of Linear Quotients}}, tackle a problem invariant theory field. 
        
    \end{itemize}
\end{frame}


\section{Installation}

\begin{frame}{Installation}
    \Huge{\centerline{\textbf{See The \href{https://www.oscar-system.org/}{Documentation}}}}
\end{frame}


%------------------------------------------------

\begin{frame}{References}
    % Beamer does not support BibTeX so references must be inserted manually as below
    \footnotesize{
        \begin{thebibliography}{99}
            \bibitem[Agostini A., Markwig, H., Noliau, C., Schleis, V., Sendra-Arranz, J., Sturmfels, B., 2022]{p1} Agostini A., Markwig, H., Noliau, C., Schleis, V., Sendra-Arranz, J., \& Sturmfels, B. (2022)
            \newblock Recovery of plane curves from branch points
            \newblock \emph{Preprint,} arXiv:2205.11287.
        \end{thebibliography}
        \begin{thebibliography}{99}
            \bibitem[Arzhantsev, I.V., \& Gaĭfullin, S.A., 2010]{p1} Arzhantsev, I.V., \& Gaĭfullin, S.A. (2010)
            \newblock Cox rings, semigroups and automorphisms of affine varieties
            \newblock \emph{Math. Sbornik,} 201(1), 3-24.
        \end{thebibliography}
        \begin{thebibliography}{99}
            \bibitem[Belotti, M., \& Panizzut, M., 2022]{p1} Belotti, M., \& Panizzut, M. (2022)
            \newblock Discrete geometry of Cox rings of blow-ups of $\mathbb{R}^3$
            \newblock \emph{Preprint,} arXiv:2208.05258.
        \end{thebibliography}
        \begin{thebibliography}{99}
            \bibitem[Berlow, K., Brandenburg, M.-C., Meroni, C., \& Shankar, I., 2022]{p1} Berlow, K., Brandenburg, M.-C., Meroni, C., \& Shankar, I. (2022)
            \newblock  Intersection bodies of polytopes
            \newblock \emph{Beitr. Algebra Geom.,} 63, 419-439.
        \end{thebibliography}
    }
\end{frame}

\begin{frame}{References}
    % Beamer does not support BibTeX so references must be inserted manually as below
    \footnotesize{
        \begin{thebibliography}{99}
            \bibitem[Breiding, P., \& Timme, S., 2018]{p1} Breiding, P., \& Timme, S. (2018)
            \newblock HomotopyContinuation.jl: A package for homotopy continuation in Julia
            \newblock \emph{Mathematical software – ICMS 2018,} pp. 458-465.
        \end{thebibliography}
        \begin{thebibliography}{99}
            \bibitem[Decker, W., Eder, C., Fieker, C., Horn, M., \& Joswig, M., 2024]{p1} Decker, W., Eder, C., Fieker, C., Horn, M., \& Joswig, M. (2024)
            \newblock The OSCAR book
            \newblock \emph{To be published} .
        \end{thebibliography}
        \begin{thebibliography}{99}
            \bibitem[Fieker, C., Hofmann, T., \& Joswig, M., 2022]{p1} Fieker, C., Hofmann, T., \& Joswig, M. (2022)
            \newblock Computing Galois groups of Ehrhart polynomials in OSCAR
            \newblock \emph{Sém. Lothar. Combin.,} 86B, 87.
        \end{thebibliography}
        \begin{thebibliography}{99}
            \bibitem[ Gardner, R.J., Koldobsky, A., \& Schlumprecht, T., 2022]{p1}  Gardner, R.J., Koldobsky, A., \& Schlumprecht, T. (2022)
            \newblock An analytic solution to the Busemann-Petty problem on sections of convex bodies
            \newblock \emph{Ann. Math.,} 149(2), 691-703.
        \end{thebibliography}
    }
\end{frame}

\begin{frame}{References}
    % Beamer does not support BibTeX so references must be inserted manually as below
    \footnotesize{
        \begin{thebibliography}{99}
            \bibitem[ Klartag, B., \& Milman, V., 2022]{p1}  Klartag, B., \& Milman, V. (2022)
            \newblock The slicing problem by Bourgain.  In A. Avila, M.T. Rassias, \& Y. Sinai (Eds.)
            \newblock \emph{Analysis at large: Dedicated to the life and work of Jean Bourgain,} (pp. 203-231). Cham, Switzerland: Springer Cham.
        \end{thebibliography}
        \begin{thebibliography}{99}
            \bibitem[Knuth, D.E., 1984]{p1} Knuth, D.E. (1984)
            \newblock  Literate programming
            \newblock \emph{Comp. J.,} 27(2), 97-111.
        \end{thebibliography}
}
\end{frame}

%------------------------------------------------

\begin{frame}
    \Huge{\centerline{\textbf{Thank You}}}
\end{frame}

%----------------------------------------------------------------------------------------
\end{document}